\documentclass[12pt]{article}
\usepackage{amsmath}
\usepackage{amssymb}
\usepackage{amsfonts}
\usepackage{rotating}
\usepackage[hidelinks]{hyperref}
\usepackage{graphicx}
\usepackage{listings}
\usepackage[T1]{fontenc}
\usepackage{textcomp}
\usepackage{lmodern}
\usepackage[colorinlistoftodos,prependcaption,textsize=tiny]{todonotes}

\title{Notes on MITRE 2.0: Predicting events}
\author{Eli Bogart}
\date{\today}
\begin{document}
\maketitle
\lstset{
basicstyle=\small, 
stringstyle=\ttfamily,
language={},
showstringspaces=false,
frame=single,
upquote=True,
breaklines=True,
breakatwhitespace=True,
numbers=left
} 
%\listoffigures
%\listoftables
\section{Introduction}
Recall that MITRE attempts to classify hosts into two groups, or
equivalently, to predict a time-independent binary host status
variable $y$, based on observations of each host's microbiome, which
may be taken at different times for each host, but which all occur in
a window $[0,T_\text{experiment}]$. If $y$ represents the occurence of
some event, we can use this approach to `predict' that event in the
ordinary sense of the word so long as the event can never occur before
$T_\text{experiment}$, and we do not care exactly \textit{when} the
event occurs after that point.

More generally, however, we will want to predict when events will
occur, and our outcome data will not be simply binary but, in effect,
a set of events, the times at which they occurred, and the subjects
for whom they occurred: $\mathbf E = \{(T_1, s_1), (T_2, s_2), \ldots,
(T_{n_\text{events}}, S_{n_\text{events}})\}$.

As an example of particular and immediate interest, we want to analyze
the precision medicine \textit{C. difficle} recurrence study. Subjects
who had just concluded treatment for an initial \textit{C. diff}
infection provided stool samples at baseline (week 0) and weekly for
up to 6 weeks thereafter, unless the infection recurred, in which case
the recurrence was noted and sample collection stopped. We are
interested in how each subject's microbiome composition over time
affects their risk of recurrence.

This is a survival analysis problem, standard if slightly complicated
because of the time-dependence of the microbiome covariates. It is
possible to imagine grafting any number of traditional methods for
survival analysis onto the MITRE framework. We will pursue a simple
approach of discretizing the problem-- here, by week-- and predicting
the occurence of an event for each subject in each week conditional on
the data in the preceding $\mathcal L$ weeks through a logistic
regression model.

\section{Notation}
\subsection{Standard MITRE notation}\label{standard}

Recall that MITRE's input (after any phylogenetic aggregation
has been done) consists of measurements $x_{os}(t_{s1}),
x_{os}(t_{s2}), \ldots$ of variables $o \in
\{1,\ldots,n_\text{variables}\}$ in samples from subjects
$s\in\{1,\ldots,n_\text{subjects}\}$ at observation timepoints
$t_{s1},t_{s2},\ldots$, which may vary across subjects, but are
assumed to all occur in the window $[0,T_\text{experiment}]$.  We
denote the total number of observations for subject $s$ as $N_{s}$ and
write $x_{osk}=x_{os}(t_{sk})$, and consider a binary outcome $y_s$
for each host.

\subsection{New notation}
Because we will translate our actual data for the survival problem
into an effective set of MITRE data, we introduce new notation for the
raw, un-transformed data. Suppose we have data from $N_\zeta$ subjects
in total, such that the observation times for subject $\zeta$ are
$\eta_{\zeta1},\eta_{\zeta2}, \ldots$ (with $H_\text{min} <
\eta_{\zeta q} < H_\text{max}$ for all $\zeta$ and $q$,) and let
$\xi_{o\zeta q}$ be the measured value of variable $o$ in subject
$\zeta$ at time $\eta_{\zeta q}$.

As above, we have a set of observed events, which we write
as ordered pairs of event times and subject indices:

\[ \mathbf E = \{(H_1, z_1), (H_2, z_2), \ldots,
(H_{n_\text{events}}, z_{n_\text{events}})\}\]

In the \textit{C. diff} case, there will be at most one event
per subject, but we can treat the more general case at no cost in
additional complexity.

Suppose also that we have for each subject $\zeta$ a censoring time
$\kappa_\zeta$ after which the subject does not participate in the
study (setting $\kappa_\zeta=H_\text{max}$ for subjects who
complete the entire study.)  In the \textit{C. diff} case, the
censoring times coincide with the recurrence times, but again, we can
treat the more general case quite naturally.

\section{Reformulating the time-dependent problem as a standard MITRE problem}

\subsection{Discretization} 
Choose $N_\mathcal W$ regularly spaced timepoints $H_\text{min} <
\mathcal W_1 < \ldots < \mathcal W_{N_\mathcal W} < H_\text{max}$,
with $\mathcal W_{2} = \mathcal W_1 + \delta$, $\mathcal W_{3} =
\mathcal W_{1} + 2\delta$ for some $\delta$.  We will assume that
$\delta = (H_\text{max} - H_\text{min})/N_\mathcal W $, with $W_1 =
H_\text{min} + \delta$, etc., but this is not conceptually necessary.

In the \text{C. diff} study, we naturally let $\mathcal W_1$ = 1 week
after the baseline, $\mathcal W_2$ = 2 weeks after the baseline, and
so on until $\mathcal W_5$ = 5 weeks after the baseline.\footnote{It
  will become clear below that it doesn't matter whether we include
  $\mathcal W_6$ at 6 weeks because we don't have any data about what
  events occur after 6 weeks.}

\subsection{Reformulation}

Conceptually, what we will do is straightforward. We define a
backwards interval $\mathcal L$ and a forward interval $\mathcal F$.
For each subject $\zeta$, for each discretized time point $W_j$, we
create an `effective subject' whose data are the data for subject
$\zeta$ between time $W_j - \mathcal L$ and time $W_j$, and whose
outcome is $y=1$ if an event occurs between time $W_j$ and time $W_j +
\mathcal F$, and $y=0$ if no event occurs in that time period.

(If no event occurs between $W_j$ and $W_j + \mathcal F$, \textit{and}
$\kappa_\zeta < W_j + \mathcal F$, we do not create the effective
subject: we don't know for sure that an event did \textit{not} happen
in the relevant time period. If $\kappa_\zeta < W_j + \mathcal F$ but
an event does occur in $(W_j, W_j + \mathcal F]$, we ignore the
  censoring and proceed: we know that at least one event occured
  before the subject left the study, and we don't care what else may
  have happened after.)

  The choice $\mathcal F=\delta$ is the most natural, as then any event
  may effect the outcome of one and only one effective subject, but it is not technically required.

Formally, we define an indicator function $I(\zeta,j)$ for $\zeta \in
\{1 \ldots N_\zeta\}$, $j \in \{1 \ldots N_\mathcal W\}$, as
\[  I(\zeta, j) = \left\{
\begin{array}{ll} 1, & | \{ (H, z) \in \mathbf E \vert z = \zeta, W_j < H \leq W_j + \mathcal F\} | > 0, \\
  1, & \kappa_\zeta \geq W_j + \mathcal F  \\
  0, & \text{otherwise}
\end{array}
\right.
\]
Now let $S = \{ (\zeta, j) \vert I (\zeta,j) = 1$ and set $n_\text{subjects}=|S|$. Order the elements of
$S$ arbitrarily (e.g., lexicographically) from $s_1$ to $s_{n_\text{subjects}}$.
For each $i \in \{1, \ldots, n_\text{subjects}\}$:
\begin{itemize}
\item Let $J(i)$ be the value
  of $j$ in $s_i$ and likewise let $Z(i)$ be the value of $\zeta$ in $s_i$.
\item Let 
\[  y_i = \left\{
\begin{array}{ll} 1, & | \{ (H, z) \in \mathbf E \vert z = Z(i), W_{J(i)} < H \leq W_{J(i)} + \mathcal F\} | > 0, \\
  0, & otherwise  \\
\end{array}
\right.
\]
\item Define
  \[ Q = \{ q | \eta_{Z(i)q} \in [W_{J(i)} - \mathcal L, W_{J(i)}]\} \]
  and let $n_{s_i}=|Q|$. Order the elements of $Q$ naturally from $q_1$ to
  $q_{n_{s_i}}$. For $1<k<n_{s_i}$, let
  \[t_{{s_i}k} = \eta_{Z(i)q_k} - (W_{J(i)} - \mathcal L)\]
\item For $1<k<n_{s_i}$, let \[x_{o{s_i}k} = \xi_{oZ(i){q_k}}\].
\end{itemize}

By construction, we have obtained a new set of MITRE input data in the
standard form shown in section \ref{standard}, with
$T_\text{experiment}=\mathcal L$. When we build a population of
detectors for this data set, a detector with time window $(a,b)$ will
effectively test whether a condition applies from time $(\mathcal
L-a)$ before the present to $(\mathcal L-b)$ before the present, and
the coefficient associated with a rule will describe its effect on the
log-odds that at least one event will occur between now and a time
$\mathcal F$ from now.

\section{Future direction: time-dependent coefficients}
In the present formulation, we assume that the rule list $R$ and
associated coefficients $\boldsymbol \beta$ do not themselves change
with time: we use the same rules and coefficients to predict the
occurence of events between $W_1$ and $W_1 + \mathcal F$ given data
between $W_1 - \mathcal L$ and $W_1$, as we do to predict events
between $W_{20}$ and $W_{20} + \mathcal F$ given data between $W_{20}$
and $W_{20}-\mathcal L$. Modifying the structure of the rule list over
time isn't clearly desirable and would require sweeping modifications
to the formulation and possibly a new MCMC sampler. Allowing
$\boldsymbol \beta$ to change over time is likely a better way to
account for the possibility that the effect of particular factors may
change over time, and this could be done with only modest effort. It
would involve changing the process of forming the matrix $A(R)$ in
section 1.3 of the MITRE supplementary note, as follows: rather than
introducing one column per rule in $R$, we introduce $N_\mathcal W$
columns, such that entry $i$ in the $\nu$-th column corresponding to a
rule $\rho$ is 1 if the rule is true for effective subject $i$
\textit{and} $J(i) = \nu$ and 0 otherwise. Then to each rule
$\rho_\mu$ there would correspond $N_\mathcal W$ separate coefficients
$\beta_{\mu 1}, \ldots, \beta_{\mu {N_\mathcal W}}$ giving the effects
of the rule at times $W_1, \ldots, W_{N_\mathcal W}$. We can then
enforce some model of the variation in coefficients over time by using
an appropriate block-diagonal covariance matrix $C$ for the normal prior on
$\boldsymbol \beta$ rather than the standard $\sigma^2_b I$:
\[ \boldsymbol \beta \vert R \sim N(\mathbf 0, C).\]

I have not implemented this method for two reasons:
\begin{itemize}
\item Lack of time to write and test the necessary code for generating
  $A$ in this way, and, in particular, to audit the existing code for
  steps that assume the covariance matrix for this prior has the
  $\sigma^2_b I$ structure and revise them
\item Strong belief that we do not now have, and are not likely in the
  near future (or, e.g., at the conclusion of the \textit{C. diff}
  study) to have, enough data that we could reasonably expect to learn
  time-depencies of the coefficients.
\end{itemize}

\section{Implementation of time-dependent outcomes}

To implement the approach above, the \texttt{events} branch of the MITRE git repository
introduces a new command \texttt{mitre\_events} (see \texttt{events.py}). Like the normal
\texttt{mitre} command this command parses a configuration file, preprocesses data, and sets up and samples
from a model accordingly; its behavior differs from the standard as follows:
\begin{itemize}
\item The options `outcome\_variable' and `outcome\_positive\_value' are ignored.
\item A new option `event\_file' in section `data' is required. It should
  be a CSV table without header, the first column giving subject IDs and the second event times:

\begin{lstlisting}[caption=Example events file]
  SubjectID1,3
  SubjectID5,2
\end{lstlisting}

\item A new option `censoring\_file' in section `data' is allowed. It
  should be a CSV table without header, the first column giving
  subject IDs and the second times, with at most one time per subject.
  Subjects not listed (or all subjects, if this file is not given) are
  assumed to have remained active until the end of the study.  (Note
  that in the \textit{C. diff} case, we can simply reuse the events
  file for this.)

\item New options `lookback', `lookahead', and
  `event\_discretization\_points' in section `preprocessing' are
  required; they give $\mathcal F$, $\mathcal F$, and $N_\mathcal W$
  respectively.

\item There is no `benchmarking' convenience option. Benchmarking will
  have to be set up manually (i.e., by setting up a configuration file
  that sets up the model without phylogenetic aggregation and invokes
  the comparison method crossvalidation, and comparing the results
  separately to those generated using a separate configuration file
  which apply the MITRE methods to data that has been phylogenetically
  aggregated.)

\item The option `load\_example' is no longer available, as none of the example
  problems have the requisite structure.
  
\item Preprocessing proceeds normally, except that the temporal
  resolution filter specified by `density\_filter\_n\_samples', etc.,
  is not applied. After normal preprocessing concludes, the
  transformation process outlined above is carried out to generate a
  new dataset.  \textit{Then} the temporal resolution filter is
  applied to the transformed data. (This allows us to require
  each pseudo-subject to have, e.g., at least one observation in $[0,
    \mathcal L/2]$ and one in $[\mathcal L/2, \mathcal L]$.)

\item Execution then proceeds normally. Note the output formats do
  not change; it is the responsibility of the user to translate each
  reported time window $(a,b)$ into ``time $(\mathcal L-a)$ before the
  present to $(\mathcal L-b)$ before the present''.
\end{itemize}

% \section{Non-microbiome time-dependent covariates}
% Theory
% Implementation

\end{document}
